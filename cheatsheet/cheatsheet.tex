\documentclass[10pt]{article}

\usepackage{amsmath}
\usepackage{amsfonts}
\usepackage[margin=1cm, landscape]{geometry}
\usepackage{multicol, tikz}


\usepackage{alltt,listings,showexpl}

\usepackage[czech]{babel}
\usepackage[utf8]{inputenc}

\def\nadpis#1\par{\par\bigskip\noindent \textbf{#1} \par}
\def\polozka #1;#2;{\par $#1$\hfill \texttt{#2}\par\smallskip}

\everymath{\displaystyle}
\begin{document}

\parindent 0 pt
\pagestyle{empty}

\setlength\columnsep{40pt}

\begin{multicols}3
  {\centering \textbf{WeBWorK cheatsheet}

    }

\nadpis Základní pravidla, tipy

\begin{itemize}
\itemsep=-3pt\raggedright
\item Notace je v podstatě stejná jako pro všechny bežně používané programy (MS Excel, OpenOffice, Pascal, Pyhton, Sage, R).
\item Často se nemusí psát značka pro násobení, stejně jako ji často vynecháváme v rukou psaném textu.
\item Nezáleží na mezerování, to můžeme využít ke zpřehlednění kódu.
\item Před odesláním můžete použít náhled, který zkontroluje formální správnost.
\item Pro prohlížeč Chrome existuje plugin WeBWorK MathView, který zobrazuje náhled hned při psaní.
\item V nastavení si můžete nastavi plugin pro zápis ve 2D.
\item Oddělovačem v desetinných číslech je tečka.
\item Posuzuje se numerická shoda v náhodných bodech. Není tedy důžetité například pořadí sčítanců nebo součinitelů. Výrazy musí být matematicky ekvivalentní, ale nejsou žádná další omezení na konkrétní formu zápisu.
\end{itemize}

\nadpis Když se nedaří

\begin{itemize}  \itemsep=-3pt\raggedright
\item Jsou desetinná čísla zapsána pomocí desetinné tečky?
\item 
Objevuje se v tabulce s výsledky po odeslání nějaká chybová hláška?
\item 
Je je po stisnutí tlačítka pro náhled zadávaná funkce skutečně rozpoznána stejně, jako je tvar, který se snažíte zadat?
\item 
Možná zadáváte špatný výsledek. Pokud to příklad umožňuje, vyvolejte si podobný příklad, podívejte se na řešení a toto řešení zkuste zapsat. Povedlo se?
\item 
Možná je příklad rozbitý. Použijte tlačítko "Email WeBWorK TA". Adresát uvidí Vaši verzi příkladu a co se snažíte zadávat. Stačí proto pouze stručně popsat problém.
\item Skvělé místo na sdílení problémů je MS Teams a k tomu určené vlákno v našem předmětu.
\end{itemize}
\vfill \null
\columnbreak

\nadpis Aritmetické operace

\polozka 7+4;7+4;
\polozka 27-4;27-4;
\polozka 7\times 4;7*4;
\polozka 73\div 44;73/44;
\polozka x^{12};x\^{ }12;
\polozka x^{12};x**12;

\nadpis Předdefinované konstanty

\polozka \pi;pi;
\polozka \frac 43 \pi r^3;  4/3 pi r\^{ }3;
\polozka e;e;
\polozka e^{kT};e\^{ }(k*T);


\nadpis Priorita operací

\polozka 4(2x^3-12);4*(2*x\^{ }3-12);
\polozka \frac{x^2-3}{3x-1};(x\^{ }2-3)/(3*x-1);
\polozka \frac{1}{(5x-1)^3};(5*x-1)\^{ }(-3);
\polozka \frac{1}{(5x-1)^3};1/((5*x-1)\^{ }(3));

\nadpis Odmocniny

\polozka \sqrt{x};sqrt(x);
\polozka \sqrt{x};x\^{ }(1/2);
\polozka \sqrt{x};x**(1/2);
\polozka \sqrt{x^2-1};sqrt(x\^{ }2-1);
\polozka \sqrt{x^2-1};(x\^{ }2-1)\^{ }(1/2);
\polozka \sqrt[3]{x^2-1};(x\^{ }2-1)\^{ }(1/3);


\nadpis Funkce

\polozka \sin(x);sin(x);
\polozka \cos(x);cos(x);
\polozka \ln(x);ln(x);
\polozka e^x;e\^{ }x;
\polozka e^x;e**x;
\polozka e^x;exp(x);

\columnbreak

\nadpis Derivace

V zadání by měl být instrukce, zda derivaci zapisovat pomocí čárky nebo jako podíl diferenciálů.

\polozka \frac{\mathrm dr}{\mathrm dt};dr/dt;
\polozka 4 \pi r ^2 \frac{\mathrm dr}{\mathrm dt};4 pi r\^{ }2 dr/dt;

\nadpis Vektory

Zapisujeme pomocí ijk-notace nebo pomocí ostrých závorek

\polozka (3,4,-1);< 3 , 4 , -1 >;
\polozka (3,4,-1);3i + 4j - k;
\polozka (x+1,4x^3);(x+1)*i + 4 x\^{ }3*j;
\polozka (x+1,4x^3);< x+1 , 4 x\^{ }3 >;

\nadpis Desetinná čísla

Oddělovačem je tečka!

\polozka 3{,}14; 3.14;
\polozka 1{,}3^{51{,}12}; (1.3)\^{ }(51.12);
\polozka 1{,}3^{51{,}12}; (1.3)**(51.12);


\nadpis Ukázky

\polozka 6kh^5\frac{\mathrm dh}{\mathrm dt};6 k h\^{ }5 dh/dt;
\polozka 23+5(m-2);23+5*(m-2);
\polozka \lambda^2-6\lambda+12;lambda\^{ }2-6lambda+12;
\polozka \frac 1{\sqrt{1-\frac {v^2}{c^2}}};(1-v\^{ }2/c\^{ }2)\^{ }(-1/2);
\polozka \frac 1{\sqrt{1-\frac {v^2}{c^2}}};1/sqrt(1-v\^{ }2/c\^{ }2);

\columnbreak

\nadpis Slovní odpovědi a \LaTeX{}

\begin{itemize}\itemsep=0pt\raggedright
\item Každý matematický výraz, číslo, proměnnou zapisujeme v matematickém prostředí. Matematické výrazy se zapisují ve značkovacím (programovacím) jazyce \LaTeX. Běžný text se zapisuje bez formátovacích značek (nejsou řezy písma, zvýrazňování atd.)
\item Matematické prostředí v řádku vyznačujeme \verb|\( ... \)|.
\item Matematické prostředí na samostatném řádku vyznačujeme \verb|\[ ... \]|.
\item Konce řádků nerozhodují. 
\item Mezery si program řídí sám. Více mezer za sebou jsou ekvivaletní s jednou mezerou.
\item Prázdný řádek odděluje odstavce.
\item Vzorce zapisujeme pomocí smluvených značek a příkazů. Používají se jenom znaky dostupné na anglické klávesnici. 
\item Znaky, které neodpovídají písmenkům anglické abecedy a formátovací znaky se vkládají pomocí příkazů. Příkazy začínají zpětným lomítkem. Působení příkazů se omezuje na jeden znak nebo na skupinu ohraničenou složenými závorkami. 
\item Program \LaTeX{} je velmi komplexní značkovací (programovací) jazyk, my využijeme jenom jeho část zaměřenou na zápis matematických výrazů. Neděste se sáhodlouhých příruček nebo učebnic tohoto jazyka. Vůbec je nebudeme potřebovat.
\item Během editace v programu WeBWorK se zobrazuje náhled výsledného vzorce. Tlačítka nad editorem usnadňují zadávání často potřebných konstrukcí bez nutnosti přepínat na anglickou klávenisci.
\end{itemize}

\nadpis Zlomky a derivace

\def\polozka#1;{\smallskip\par $#1$\hfill}

\polozka \frac\pi2;\verb|\frac \pi 2|
\polozka \frac{x+2}{3x-1};\verb|\frac {x+2} {3x-1}|
\polozka \frac{\mathrm dx}{\mathrm dt};\verb|\frac{\mathrm dx}{\mathrm dt}|
\polozka \frac{\mathrm d^2x}{\mathrm dt^2};\verb|\frac{\mathrm d^2x}{\mathrm dt^2}|
\polozka \frac{\partial u}{\partial x};\verb|\frac{\partial u}{\partial x}|
\polozka \frac{\partial^2 u}{\partial x^2};\verb|\frac{\partial u^2}{\partial x^2}|

\nadpis Lineární algebra

\polozka 2\vec e_1 -4\vec e_2;\verb|2\vec e_1 -4\vec e_2|
\polozka \begin{pmatrix}1&2\\x&y^3\end{pmatrix};{ }
\polozka { };\verb|\begin{pmatrix}1&2\\x&y^3\end{pmatrix}|


\nadpis Mocniny a odmocniny

\polozka \sqrt 3;\verb|\sqrt 3|
\polozka \sqrt {31};\verb|\sqrt {31}|
\polozka \sqrt{x^{12}-\pi};\verb|\sqrt{x^{12}-\pi}|
\polozka -k(T-T_0);\verb|-k(T-T_0)|
\polozka \left(1-\frac xK\right);\verb|\left(1-\frac xK\right)|

\nadpis Písmena řecké abecedy

\noindent
\foreach \i in {alpha,beta, gamma, pi, omega, delta, varphi, psi, Omega, Pi, Phi, Delta}
{$\expandafter \csname\i\endcsname $ \hbox{\texttt{\textbackslash\i}}, }


\nadpis Vektorová anaýza

\polozka \nabla f;\verb|\nabla f|
\polozka \nabla\cdot\vec F;\verb|\nabla\cdot\vec F|
\polozka \nabla\times\vec F;\verb|\nabla\times\vec F|
\polozka \oint\vec F\,\mathrm d\vec r;\verb|\oint\vec F\,\mathrm d\vec r|

\nadpis Funkce

\polozka e^{2x-1};\verb|e^{2x-1}|
\polozka \sin(2x-1);\verb|\sin(2x-1)|
\polozka \cos(2x-\pi);\verb|\cos(2x-\pi)|
\polozka \ln(2x-1);\verb|\ln(2x-1)|

\nadpis Nerovnosti

U znaménka ostře menší musí následovat mezera, jinak html prohlížeč tento znak
interpretuje jako otevření html tagu.

\polozka a\leq x\leq\infty;\verb|a\leq x\leq\infty|
\polozka a\geq x\geq0;\verb|a\geq x\geq0|
\polozka a < x < b;\verb|a < x < b|
\polozka a>x>b;\verb|a>x>b|

\nadpis Další

\polozka \pm 1;\verb|\pm 1|
\polozka \int_0^{\frac \pi 2} x\,\mathrm dx;

\null\hfill\verb|\int_0^{\frac\pi2} x\,\mathrm dx|
\polozka \int_0^{\frac \pi 2} x dx;\verb|\int_0^{\frac\pi2} x dx|

V přednáškách nebo na Wikipedii si najděte vzorec, u kterého chcete vidět zdrojový kód. Poté klikněte pravým tlčítkem a vyberte v menu Show Math As a TeX Commands.


\end{multicols}

\hrule

\begin{multicols}2

\nadpis Tlačítka u editačního pole ve WeBWorK

Tlačítka vkládají text napsaný na tlačítku. Pokud je označen blok, je text XXX nahrazen tímto blokem. 

Níže je vždy výchozí text, černě je zvýrazněn označený text v editoru před stisknutím tlačítka, dále je efekt po stisknutí tlačítka a výsledná sazba


\verb|x1,2|
\hfill
\texttt{x\colorbox{black}{\color{white}1,2}}
\hfill
\verb|x_{1,2}| \hfill $x_{1,2}$

\verb|2x3|
\hfill
\texttt{2x\colorbox{black}{\color{white}3}}
\hfill
\verb|2x^{3}| \hfill $2x^{3}$

Tlačítko pro vložení zlomku se snaží v označeném textu najít první lomítko a podle něj určí čitatel a jmenovatel. Je to čistě textová operace, řídí se hranicemi označeného textu, neřídí se matematickými pravidly ani pravidly systému \LaTeX. Je na uživateli, aby postup práce přizpůsobil očekávanému výsledku.



\def\polozka{\par}

\polozka \texttt{1+x/K}
\hfill \texttt{1+\colorbox{black}{\color{white}x/K}}\hfill \verb.1+\frac{x}{K}. \hfill $1+\frac xK$

\polozka \texttt{1+x/K}
\hfill \texttt{\colorbox{black}{\color{white}1+x/K}}\hfill \verb.\frac{1+x}{K}. \hfill $\frac {1+x}K$


\end{multicols}

\newpage
%\nadpis Ukázky \LaTeX u

\lstset{
  literate={á}{{\'a}}1
           {ú}{{\'u}}1
           {ů}{{\accent23 u}}1
           {í}{{\'i}}1
           {é}{{\'e}}1
           {ý}{{\'y}}1
           {ř}{{\v r}}1
           {č}{{\v c}}1
           {š}{{\v s}}1
           {ž}{{\v z}}1
           {ě}{{\v e}}1
}

\lstset{language=TeX}
\lstset{basicstyle=\ttfamily,breaklines=true}

% Vlevo je text a vpravo kód, který tento text produkuje. Na zalomení řádků a počtu mezer v kódu nezáleží. Zalomení řádků a počet mezer volíme tak, aby byl kód přehledný.


\hrule
\begin{multicols}2

  \begin{minipage}{1.0\linewidth}
    \parindent=2em

Logistická rovnice je rovnice
\[
\frac{\mathrm dx}{\mathrm dt} =
r x \left(1-\frac{x}{K}\right),
\]
kde \(x\) je velikost populace, 
\(r\) je konstanta úměrosti a  
\(K\) je nosná  kapacita prostředí.

Pro \(x > K\) je řešení klesající 
a pro \(0 < x< K\) rostoucí.

  \end{minipage}

\columnbreak
  
\begin{lstlisting}
Logistická rovnice je rovnice
\[      \frac{\mathrm dx}{\mathrm dt} = 
        r x \left(1-\frac{x}{K}\right),      \]
kde \(x\) je velikost populace, \(r\) je konstanta 
úměrosti a \(K\) je nosná  kapacita prostředí.

Pro \(x>K\) je řešení klesající a pro \(0 < x < K\) 
rostoucí.
\end{lstlisting}

\end{multicols}

\hrule

\begin{multicols}2

      \begin{minipage}{1.0\linewidth}
    \parindent=2em

      Model, který vyjadřuje, že teplota tekutiny, klesá 
rychlostí úměrnou teplotnímu rozdílu mezi teplotou 
tekutiny a teplotou okolí, je
\[ \frac{\mathrm dT}{\mathrm dt}=-k(T-T_0),\]
kde \(T\) je teplota tekutiny, \(T_0\) je teplota 
okolí a \(k\) je konstanta.

Druhý model, který popisuje situaci, kdy do tekutiny 
navíc ponoříme ohřívač přispívající k růstu teploty 
konstantní rychlostí je 
\[\frac{\mathrm dT}{\mathrm dt}=-k(T-T_0)+q,\] 
kde \(q\) je konstantní rychlost s jakou přispívá 
ohřívač k růstu teploty.

Oba modely mají stabilní konstantní řešení a to 
\[T=T_0\] v případě prvního modelu a 
\[T=T_0+\frac qk\] v případě druhého modelu.
  \end{minipage}


\begin{lstlisting}
Model, který vyjadřuje, že teplota tekutiny, klesá 
rychlostí úměrnou teplotnímu rozdílu mezi teplotou 
tekutiny a teplotou okolí, je
\[ \frac{\mathrm dT}{\mathrm dt}=-k(T-T_0),\]
kde \(T\) je teplota tekutiny, \(T_0\) je teplota 
okolí a \(k\) je konstanta.

Druhý model, který popisuje situaci, kdy do tekutiny 
navíc ponoříme ohřívač přispívající k růstu teploty 
konstantní rychlostí je 
\[\frac{\mathrm dT}{\mathrm dt}=-k(T-T_0)+q,\] 
kde \(q\) je konstantní rychlost s jakou přispívá 
ohřívač k růstu teploty.

Oba modely mají stabilní konstantní řešení a to 
\[T=T_0\] 
v případě prvního modelu a 
\[T=T_0+\frac qk\] 
v případě druhého modelu.
\end{lstlisting}


  
\end{multicols}

\newpage

\begin{multicols}{2}
  \begin{minipage}{1.0\linewidth}
    \parindent=2em
    Rychlost stoupání je derivace nadmořské výšky podle času. 
Rychlost růstu počtu obyvatel je derivace počtu obyvatel podle času.
Podle zadání je \(\frac{\mathrm dh}{\mathrm dt}=0.2\,\mathrm{m}/\mathrm{rok} \)
a \(\frac{\mathrm dN}{\mathrm dt}=100\,\mathrm{obyvatel}/\mathrm{rok}. \)

Derivováním vztahu \(S=\pi r^2\) pro obsah kruhu dostáváme
\[\frac{\mathrm dS}{\mathrm dt}= \frac{\mathrm dS}{\mathrm dr} \frac{\mathrm dr}{\mathrm dt}=2\pi r  \frac{\mathrm dr}{\mathrm dt}.\] Po dosazení zadaných hodnot \(r=9000\,\mathrm{m}\) a \(\frac{\mathrm dr}{\mathrm dt}=0.2\,\mathrm{m}/\mathrm{rok}\) dostáváme
\[\frac{\mathrm dS}{\mathrm dt} =3600 \pi\, \mathrm{m}^2/\mathrm{rok}.\]

  \end{minipage}

\columnbreak
  
\begin{lstlisting}
Rychlost stoupání je derivace nadmořské výšky podle 
času. Rychlost růstu počtu obyvatel je derivace počtu 
obyvatel podle času. Podle zadání je 
\(\frac{\mathrm dh}{\mathrm dt}=
   0.2\,\mathrm{m}/\mathrm{rok} \)
a \(\frac{\mathrm dN}{\mathrm dt}=
    100\,\mathrm{obyvatel}/\mathrm{rok}. \)

Derivováním vztahu \(S=\pi r^2\) pro obsah kruhu 
dostáváme
\[\frac{\mathrm dS}{\mathrm dt} = 
   \frac{\mathrm dS}{\mathrm dr} 
   \frac{\mathrm dr}{\mathrm dt} 
   = 2\pi r  \frac{\mathrm dr}{\mathrm dt}.      \] 
Po dosazení zadaných hodnot \(r=9000\,\mathrm{m}\) a 
\(\frac{\mathrm dr}{\mathrm dt}
   =0.2\,\mathrm{m}/\mathrm{rok}\) dostáváme
\[
\frac{\mathrm dS}{\mathrm dt} =
  3600 \pi\, \mathrm{m}^2/\mathrm{rok}.    \]
\end{lstlisting}

\end{multicols}


S jistou mírou velkorysosti může pro začátečníka být přdchozí text zjednodušen takto.
(Jednotky jsou zapsány textově, nejsou odděleny od hodnoty mezerou správné velikosti podle normy a v podílu diferenciálů nezapínáme textový
režim pro písmeno d.)

\begin{multicols}{2}
  \begin{minipage}{1.0\linewidth}
    \parindent=2em
Rychlost stoupání je derivace nadmořské výšky podle času. 
Rychlost růstu počtu obyvatel je derivace počtu obyvatel podle času.
Podle zadání je \(\frac{dh}{dt}=0.2\) m/rok
a \(\frac{dN}{dt}=100\)obyvatel/rok. 

Derivováním vztahu \(S=\pi r^2\) pro obsah kruhu dostáváme
\[\frac{dS}{dt}= \frac{dS}{dr} \frac{dr}{dt}=2\pi r  \frac{dr}{dt}.\]
Po dosazení zadaných hodnot 
\(r=9000\)m a \(\frac{dr}{dt}=0.2\)m/rok 
dostáváme \(\frac{dS}{dt} = 3600 \pi\) 
m\({}^2\)/rok.

  \end{minipage}

\columnbreak
  
\begin{lstlisting}
Rychlost stoupání je derivace nadmořské výšky podle 
času. Rychlost růstu počtu obyvatel je derivace počtu 
obyvatel podle času. Podle zadání je 
\(\frac{dh}{dt}=0.2\) m/rok
a \(\frac{dN}{dt}=100\)obyvatel/rok. 

Derivováním vztahu \(S=\pi r^2\) pro obsah kruhu 
dostáváme
\[\frac{dS}{dt}= \frac{dS}{dr} \frac{dr}{dt}
      = 2\pi r  \frac{dr}{dt}.\]
Po dosazení zadaných hodnot 
\(r=9000\)m a \(\frac{dr}{dt}=0.2\)m/rok 
dostáváme \(\frac{dS}{dt} = 3600 \pi\) 
m\({}^2\)/rok.

\end{lstlisting}

\end{multicols}



\end{document}
